\documentclass[conference]{IMSTemplate}

\input epsf
\usepackage{graphicx}
\usepackage{xcolor}

\hyphenation{op-tical net-works semi-conduc-tor IEEEtran}

\usepackage{fancyhdr}
%\SetKwComment{Comment}{/* }{ */}
\usepackage{algorithm}
\usepackage{algpseudocode}
\usepackage{cuted}
\usepackage{flushend}
\usepackage{amsmath}
\usepackage{amssymb}
\usepackage{mathrsfs}




%*********
%\usepackage[sorting = none, backend = bibtex, style=numeric-comp]{biblatex}
%\usepackage{cite}
%\addbibresource{biblatex-examples.bib}
%*******

%\pagestyle{fancy}
\begin{document}

% paper title
%\title{Submission Format for IPVC-CyberSec21 (Title in 24-point Times font)}
% If the \LARGE is deleted, the title font defaults to  24-point.
% Actually, 
% the \LARGE sets the title at 17 pt, which is close enough to 18-point.
%+++++++++++++++++++++++++++++++++++++++++++
\title{\LARGE Machine Learning for User Mobility Management in a Mobile Fog Computing Environment}

\author{\authorblockN{Hamza ELHAOU\textsuperscript{1}}
\authorblockA{Information Processing and Decision Support Laboratory \\
	University of Sultan Moulay Slimane. B.P. 523, 23000 Beni Mellal, Morocco }
\textsuperscript{1}hamza.elhaou@usms.ma}

%
%Atlanta, Georgia 30332--0250\\
%Email: mshell@ece.gatech.edu}
%\authorblockA{Starfleet Academy\\ 
%San Francisco, California 96678-2391\\
%Fax: (888) 555--1212}}
%
%Laboratory of Innovation in Mathematics, Applications and Information Technologies (LIMATI), Polydisciplinary Faculty, Sultan
%Moulay Slimane University, Po. Box 592, 23000 Beni Mellal, Morocco

% use only for invited papers
%\specialpapernotice{(Invited Paper)}

% make the title area
\maketitle

\begin{abstract}
This article focuses on minimizing energy consumption in a fog computing environment through the use of machine learning methods for enhanced classification of mobile users. We integrated the clustering algorithms K-means and Self-Organizing Maps (SOM) to improve performance. With the proliferation of connected devices and cloud-based applications, the exponential increase in data generation requires effective solutions. Fog computing, bringing computational capabilities closer to the network edge, emerges as a promising solution despite the challenge of high energy consumption. Our results prove superior performance with SOM compared to K-means, and better outcomes with K-means compared to the ifogsim method. This study underscores the efficiency of machine learning methods in minimizing energy consumption and enhancing the classification of mobile users in a fog computing environment
\\

\end{abstract}



\section{Introduction}  

The rapid evolution of information and communication technologies has led to an exponential increase in data generated by connected devices. Fog computing emerges as a promising solution for handling this vast amount of data by bringing computation and storage closer to the network edge. However, this approach can result in high energy consumption, necessitating optimization to ensure efficient resource usage.

Fog computing is a data processing model aiming to shift certain computation and storage functionalities from the central cloud to devices at the network's edge, such as routers, wireless access points, and IoT devices. This reduces data transfer delays to the central cloud, enabling quicker decision-making near connected devices.

Nevertheless, the fog computing approach can lead to high energy consumption. Edge devices used for data processing and storage require energy to operate. The proliferation of connected devices and the processing of large amounts of data can significantly increase overall energy consumption in the fog environment \cite{ref1}.

Therefore, to ensure efficient resource usage and minimize energy consumption, optimization is needed in the design and implementation of fog infrastructures. This may include techniques such as intelligent energy management, the use of energy-efficient hardware, resource consolidation, and optimization of data processing algorithms \cite{ref11}.

The primary goal of this article is to present an innovative approach to minimize energy consumption in a fog computing environment. To achieve this, we employed the Ifogsim simulation framework, providing a flexible and realistic modeling platform to assess the performance of fog computing systems. By integrating the clustering algorithms K-means and SOM, we aim to enhance energy efficiency and reduce operational costs. The selection of these clustering algorithms is based on their proven capabilities to efficiently identify patterns and form groups.

In this article, we will detail the implementation of clustering algorithms within the Ifogsim simulation framework. We will also explain the methodology used to evaluate performance in terms of energy consumption and other relevant metrics. The obtained results will be analyzed and compared with those of other existing approaches to demonstrate the effectiveness of our proposed approach \cite{ref12}.

This article offers an innovative perspective on minimizing energy consumption in a fog computing environment using Ifogsim and integrating clustering algorithms K-means and SOM.
\\


\section{Fog Computing}
The term "Fog Computing" was coined by Cisco and represents an extension of the Cloud Computing paradigm. The principle of Fog Computing is to move computing, networking, and storage equipment to the edge of the network, closer to the end nodes of the Internet of Things (IoT) as shown in Figure \ref{fog}. This intermediate layer allows for bringing computing equipment closer to the data source, thereby providing better efficiency and reducing latency. In essence, Fog Computing serves as an extension of the Cloud layer, enabling more potent management of computing resources.
Fog Computing is a data processing model that relies on the decentralization of data storage and processing. Unlike Cloud Computing, where data is stored and processed in centralized data centers, Fog Computing allows for data processing at the network edge, closer to its source. This processing model reduces latency and bandwidth usage by moving data processing closer to the original sources, thereby reducing the load on central networks \cite{ref2}.
Fog Computing can be applied in various domains, including smart cities, healthcare, autonomous vehicles, and manufacturing industries, where significant amounts of data are generated by connected devices such as IoT sensors, smartphones, and laptops. By utilizing Fog Computing, data can be processed in real-time, improving data processing efficiency and speed.
Additionally, Fog Computing improves data security by reducing the risk of cyber-attacks. Distributing data across multiple edge devices reduces data vulnerability to attacks and network failures \cite{ref13}.
Fog nodes represent edge devices that can be placed anywhere with network connectivity, such as along railways, traffic controllers, parking meters, and other locations. Fog does not replace Cloud Computing but complements it. It improves latency and addresses security concerns when transmitting data to the Cloud. By closely connecting to end devices, Fog Computing also improves overall system efficiency by increasing the performance of key cyber-physical systems. In summary, Fog nodes play a crucial role in optimizing IoT systems by enabling faster and more secure communication between devices and Cloud Computing \cite{ref20}.
\begin{figure}[H]
  \begin{center}
   \includegraphics[width=0.4\textwidth] {fog.png} 
   \caption{ Distributed Data Processing in a Fog Computing Environment }
   \label{fog}
  \end{center}
  \end{figure} 
The Fog environment \ref{fog} is a concept that has emerged in response to the needs for real-time data processing, mobility, and connectivity in IoT environments, particularly in urban settings. This distributed computing architecture, as shown in Figure 1.2, relies on the use of Fog nodes to provide decentralized computing, storage, and networking services to connected objects. The key components of the Fog computing environment are as follows:

\begin{itemize}
\item Cloud servers: These are remote data centers that provide storage, processing, and data analysis services. Cloud servers are used to process data collected by nodes and proxy servers, enabling more accurate and comprehensive insights \cite{ref4}.
\item  Proxy servers: These servers act as intermediaries between nodes and Cloud servers. Proxy servers optimize bandwidth, accelerate data transmission, and protect data against attacks.
\item Nodes (or edge nodes): These are devices located at the network edge, such as routers, switches, sensors, monitoring cameras, and connected objects, etc. Nodes are used to collect, store, and process data, reducing network traffic and improving responsiveness.
\item IoT devices: These are Internet-connected devices that collect data from sensors or other sources and transmit it to nodes or Cloud servers for processing. IoT devices are used in various applications, such as monitoring, home automation, healthcare, etc \cite{ref15}.
\end{itemize}












\section{Conclusion}

This Article emphasizes the importance of energy optimization in fog computing and presents an innovative approach using the Ifogsim framework and K-means/SOM clustering algorithms. The results demonstrate improved energy efficiency and overall system performance. Ifogsim enables realistic evaluation, considering node interactions, communication, and energy consumption. Integration of K-means and SOM algorithms identifies efficient clustering patterns, reducing energy consumption and communication distances. The approach outperforms traditional methods, maintaining acceptable latency and quality of service. Further research can explore alternative clustering algorithms and decision-making mechanisms for enhanced energy optimization. This approach contributes to sustainable resource utilization and offers valuable insights for fog computing professionals and researchers.

\bigskip

\bigskip
\bigskip
\bigskip
\bigskip
\bigskip
\begin{thebibliography}{1}
\bibitem{ref1} ing Qian, Zhiguo Luo, Yujian Du, and Leitao Guo. Cloud computing : An overview. In Martin Gilje Jaatun, Gansen Zhao, and Chunming Rong, editors, Cloud Computing, Lecture Notes in Computer Science, pages 626–631. Springer computing : état de l’art.


\bibitem{ref2} Fatima Zahra Fagroud, El Habib Benlahmar, Sanaa Elfilali, and Hicham Toumi. IOT et cloud computing : état de l’art..

\bibitem {ref3} parna Kumari, Sudeep Tanwar, Sudhanshu Tyagi, and Neeraj Kumar. Fog computing for healthcare 4.0 environment : Opportunities and challenges. 72 :1–13

\bibitem {ref4} ajkumar Buyya and Satish Narayana Srirama. Fog and Edge Computing : Principles and Paradigms. John Wiley and  Sons.


\bibitem {ref5}  Mohammed Islam Naas, Jalil Boukhobza, Philippe Raipin Parvedy, and Laurent Lemarchand. An extension to iFogSim to enable the design of data placement strategies. In 2018 IEEE 2nd International Conference on Fog and Edge Computing (ICFEC), pages 1–8.




\end{thebibliography}


\end{document}